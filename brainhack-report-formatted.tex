%% BioMed_Central_Tex_Template_v1.06
%%                                      %
%  bmc_article.tex            ver: 1.06 %
%                                       %

%%IMPORTANT: do not delete the first line of this template
%%It must be present to enable the BMC Submission system to
%%recognise this template!!

%%%%%%%%%%%%%%%%%%%%%%%%%%%%%%%%%%%%%%%%%
%%                                     %%
%%  LaTeX template for BioMed Central  %%
%%     journal article submissions     %%
%%                                     %%
%%          <8 June 2012>              %%
%%                                     %%
%%                                     %%
%%%%%%%%%%%%%%%%%%%%%%%%%%%%%%%%%%%%%%%%%


%%%%%%%%%%%%%%%%%%%%%%%%%%%%%%%%%%%%%%%%%%%%%%%%%%%%%%%%%%%%%%%%%%%%%
%%                                                                 %%
%% For instructions on how to fill out this Tex template           %%
%% document please refer to Readme.html and the instructions for   %%
%% authors page on the biomed central website                      %%
%% http://www.biomedcentral.com/info/authors/                      %%
%%                                                                 %%
%% Please do not use \input{...} to include other tex files.       %%
%% Submit your LaTeX manuscript as one .tex document.              %%
%%                                                                 %%
%% All additional figures and files should be attached             %%
%% separately and not embedded in the \TeX\ document itself.       %%
%%                                                                 %%
%% BioMed Central currently use the MikTex distribution of         %%
%% TeX for Windows) of TeX and LaTeX.  This is available from      %%
%% http://www.miktex.org                                           %%
%%                                                                 %%
%%%%%%%%%%%%%%%%%%%%%%%%%%%%%%%%%%%%%%%%%%%%%%%%%%%%%%%%%%%%%%%%%%%%%

%%% additional documentclass options:
%  [doublespacing]
%  [linenumbers]   - put the line numbers on margins

%%% loading packages, author definitions

\documentclass[twocolumn]{bmcart}% uncomment this for twocolumn layout and comment line below
%\documentclass{bmcart}

%%% Load packages
\usepackage{amsthm,amsmath}
\usepackage{siunitx}
\usepackage{mfirstuc}
%\RequirePackage{natbib}
\usepackage[colorinlistoftodos]{todonotes}
\RequirePackage{hyperref}
\usepackage[utf8]{inputenc} %unicode support
%\usepackage[applemac]{inputenc} %applemac support if unicode package fails
%\usepackage[latin1]{inputenc} %UNIX support if unicode package fails
\usepackage[htt]{hyphenat}

\usepackage{array}
\newcolumntype{L}[1]{>{\raggedright\let\newline\\\arraybackslash\hspace{0pt}}p{#1}}

%%%%%%%%%%%%%%%%%%%%%%%%%%%%%%%%%%%%%%%%%%%%%%%%%
%%                                             %%
%%  If you wish to display your graphics for   %%
%%  your own use using includegraphic or       %%
%%  includegraphics, then comment out the      %%
%%  following two lines of code.               %%
%%  NB: These line *must* be included when     %%
%%  submitting to BMC.                         %%
%%  All figure files must be submitted as      %%
%%  separate graphics through the BMC          %%
%%  submission process, not included in the    %%
%%  submitted article.                         %%
%%                                             %%
%%%%%%%%%%%%%%%%%%%%%%%%%%%%%%%%%%%%%%%%%%%%%%%%%


%\def\includegraphic{}
%\def\includegraphics{}

%%% Put your definitions there:
\startlocaldefs
\endlocaldefs


%%% Begin ...
\begin{document}

%%% Start of article front matter
\begin{frontmatter}

\begin{fmbox}
\dochead{Report from 2015 OHBM Hackathon (HI)}

%%%%%%%%%%%%%%%%%%%%%%%%%%%%%%%%%%%%%%%%%%%%%%
%%                                          %%
%% Enter the title of your article here     %%
%%                                          %%
%%%%%%%%%%%%%%%%%%%%%%%%%%%%%%%%%%%%%%%%%%%%%%

\title{Human Connectome Project Minimal Preprocessing Pipelines to Nipype}
\vskip2ex
\projectURL{Project URL: \url{https://github.com/ericearl/hcp2nipype-hack2015/}}

\author[
addressref={aff1},
corref={aff1},
email={earl@ohsu.edu}
]{\inits{EE} \fnm{Eric} \snm{Earl}}
\author[
addressref={aff1},
%
email={demeter@ohsu.edu}
]{\inits{DD} \fnm{Damion} \snm{Demeter}}
\author[
addressref={aff1},
%
email={millska@ohsu.edu}
]{\inits{KM} \fnm{Kate} \snm{Mills}}
\author[
addressref={aff2},
%
email={paulglad.mihai@uni-greifswald.de}
]{\inits{GM} \fnm{Glad} \snm{Mihai}}
\author[
addressref={aff3},
%
email={luka.ruzic@duke.edu}
]{\inits{GM} \fnm{Luka} \snm{Ruzic}}
\author[
addressref={aff4},
%
email={nick.ketz@gmail.com}
]{\inits{NK} \fnm{Nick} \snm{Ketz}}
\author[
addressref={aff4},
%
email={andrew.reineberg@colorado.edu}
]{\inits{AR} \fnm{Andrew} \snm{Reineberg}}
\author[
addressref={aff4},
%
email={marianne.reddan@colorado.edu}
]{\inits{MR} \fnm{Marianne} \snm{Reddan}}
\author[
addressref={aff5},
%
email={algoddings@doctors.org.uk}
]{\inits{ALG} \fnm{Anne-Lise} \snm{Goddings}}
\author[
addressref={aff6},
%
email={javier.gonzalez-castillo@nih.gov}
]{\inits{JGC} \fnm{Javier} \snm{Gonzalez-Castillo}}
\author[
addressref={aff7},
%
email={chrisgor@stanford.edu}
]{\inits{KJG} \fnm{Krzysztof J.} \snm{Gorgolewski}}

%%%%%%%%%%%%%%%%%%%%%%%%%%%%%%%%%%%%%%%%%%%%%%
%%                                          %%
%% Enter the authors' addresses here        %%
%%                                          %%
%% Repeat \address commands as much as      %%
%% required.                                %%
%%                                          %%
%%%%%%%%%%%%%%%%%%%%%%%%%%%%%%%%%%%%%%%%%%%%%%

\address[id=aff1]{%
  \orgname{Oregon Health \& Science University},
  \city{Portland},
  \street{3181 SW Sam Jackson Park Road},
  \postcode{97239},
  \postcode{Oregon},
  \cny{USA}
}
\address[id=aff2]{%
  \orgname{University of Greifswald},
  \city{Greifswald},
  \street{Domstraße 11},
  \postcode{17489},
  %
  \cny{Germany}
}
\address[id=aff3]{%
  \orgname{Duke Institute for Brain Sciences},
  \city{Durham},
  \street{308 Research Drive, LSRC M051},
  \postcode{27708},
  %
  \cny{USA}
}
\address[id=aff4]{%
  \orgname{Department of Psychology and Neuroscience, Unviersity of Colorado},
  \city{Boulder},
  \street{Muenzinger D244, 345 UCB},
  \postcode{80309},
  %
  \cny{USA}
}
\address[id=aff5]{%
  \orgname{Institute of Cognitive Neuroscience, University College London},
  \city{London},
  \street{17 Queen Square},
  \postcode{WC1N 3AR},
  %
  \cny{United Kingdom}
}
\address[id=aff6]{%
  \orgname{Section on Functional Imaging Methods, Laboratory of Brain and
Cognition, National Institute of Mental Health},
  \city{Bethedsa},
  \street{Building 10, Room 1D80, 10 Center Dr.~MSC 1148},
  \postcode{20892},
  \postcode{Maryland},
  \cny{USA}
}
\address[id=aff7]{%
  \orgname{Department of Psychology, Stanford University},
  \city{Stanford},
  \street{Jordan Hall, Building 420, 450 Serra Mall},
  \postcode{94305},
  \postcode{California},
  \cny{USA}
}

%%%%%%%%%%%%%%%%%%%%%%%%%%%%%%%%%%%%%%%%%%%%%%
%%                                          %%
%% Enter short notes here                   %%
%%                                          %%
%% Short notes will be after addresses      %%
%% on first page.                           %%
%%                                          %%
%%%%%%%%%%%%%%%%%%%%%%%%%%%%%%%%%%%%%%%%%%%%%%

\begin{artnotes}
\end{artnotes}

%\end{fmbox}% comment this for two column layout

%%%%%%%%%%%%%%%%%%%%%%%%%%%%%%%%%%%%%%%%%%%%%%
%%                                          %%
%% The Abstract begins here                 %%
%%                                          %%
%% Please refer to the Instructions for     %%
%% authors on http://www.biomedcentral.com  %%
%% and include the section headings         %%
%% accordingly for your article type.       %%
%%                                          %%
%%%%%%%%%%%%%%%%%%%%%%%%%%%%%%%%%%%%%%%%%%%%%%

%\begin{abstractbox}

%\begin{abstract} % abstract
	
%Blank Abstract

%\end{abstract}



%%%%%%%%%%%%%%%%%%%%%%%%%%%%%%%%%%%%%%%%%%%%%%
%%                                          %%
%% The keywords begin here                  %%
%%                                          %%
%% Put each keyword in separate \kwd{}.     %%
%%                                          %%
%%%%%%%%%%%%%%%%%%%%%%%%%%%%%%%%%%%%%%%%%%%%%%

%\vskip1ex

%\projectURL{\url{https://github.com/ericearl/hcp2nipype-hack2015/}}
%\projectURL{https://github.com/ericearl/hcp2nipype-hack2015/}

% MSC classifications codes, if any
%\begin{keyword}[class=AMS]
%\kwd[Primary ]{}
%\kwd{}
%\kwd[; secondary ]{}
%\end{keyword}

%\end{abstractbox}
%
\end{fmbox}% uncomment this for twcolumn layout

\end{frontmatter}

%{\sffamily\bfseries\fontsize{10}{12}\selectfont Project URL: \url{https://github.com/ericearl/hcp2nipype-hack2015/}}

%%% Import the body from pandoc formatted text
\section{Introduction}\label{introduction}

The goal was to convert the Human Connectome Project (HCP) Minimal
Preprocessing Pipelines into Nipype code.

The HCP minimal preprocessing pipelines \cite{Glasser2013} represent a
significant advance in image processing pipelines in our time. They
provide preprocessed volume and surface data in native and atlas space,
for both functional and structural data. Nipype is an open source
neuroimaging project for designing imaging pipelines which has been
around since 2011 and provides many excellent features for provenance
and reliability of processing pipelines \cite{Gorgolewski2011}.
Together, these two pieces of software would allow for a more robust,
more flexible synergy of pipeline design and operability.

\section{Approach}\label{approach}

The first goal was to train the would-be Nipype developers on the Nipype
python standards for writing and running interfaces. Once trained, the
plan was to implement the HCP scripts into Nipype interfaces from the
top-level inward to the sub-level scripts. The secondary goal was to
make these sub-level scripts more flexible and require less specific
scans to run the pipelines. The collection of nine ultimate pipelines to
implement were with or without T1s or T2s and with or without Fieldmap
or Reverse-Phase-Encode EPIs as seen in Table \ref{pipetable}.

\begin{table*}[t!]
\caption{\label{pipetable}Nine pipelines to be implemented.}
\begin{tabular}{l l l l l}
 \hline\noalign{\smallskip}
  EPI & T1 & T2 & Diffusion Field Map & Reverse Phase Encode EPI \\
    \hline\noalign{\smallskip}
  N & N & N & 1 & 0 \\
  N & N & 0 & 1 & 0 \\
  N & 0 & N & 1 & 0 \\
  N & N & N & 0 & N \\
  N & N & 0 & 0 & N \\
  N & 0 & N & 0 & N \\
  N & N & N & 0 & 0 \\
  N & N & 0 & 0 & 0 \\
  N & 0 & N & 0 & 0 \\
  \noalign{\smallskip}\hline
\end{tabular}
\end{table*}

\section{Results}\label{results}

Conceptually these goals sounded reasonable enough to do all HCP scripts
at once during the hackathon, but the learning and additional setup time
was not accounted for, so the scope of the project was too big for two
days of on and off coding, even among our eleven developers.
Distributing Nipype knowledge from two experts to nine novices over two
days was not an easy beginning task, but most of the novices had gained
knowledge of Nipype usage by the end of the hackathon. Some work began
during the hackathon converting HCP scripts into Nipype pipelines,
however not much progress was made due to the unanticipated large scope
of work. The second day, an epiphany came about that the original goal,
as stated, would have only involved making five top-level wrappers for
the five HCP top-level scripts. This also slowed some progress. The
secondary goal of generalizing the HCP scripts was discussed, but not
thoroughly explored or documented. There has only been some progress in
generalization I am aware of in the Damien Fair, PA-C, PhD, Neuroimaging
Lab at OHSU. This turnout of developers during an open hackathon is
encouraging and demonstrates the importance of trying to fuse these two
systems (Nipype and the HCP scripts) to work together. Work on the
repository halted after the hackathon, but the team is still available.

\section{Conclusions}\label{conclusions}

More work is needed to truly contribute back to the HCP
Pipelines\footnote{\url{https://github.com/Washington-University/Pipelines}}.
The greatest achievement of the hackathon project was forming a
collaborative team of interested Nipype developers who were trained and
are ready to continue collaborating across seven institutions. Future
work will continue trying to achieve the original goals as stated, but
may need an organizer to hold the team accountable to deadlines. To get
involved with this project, please contact Eric Earl, earl@ohsu.edu.

%%%%%%%%%%%%%%%%%%%%%%%%%%%%%%%%%%%%%%%%%%%%%%
%%                                          %%
%% Backmatter begins here                   %%
%%                                          %%
%%%%%%%%%%%%%%%%%%%%%%%%%%%%%%%%%%%%%%%%%%%%%%

\begin{backmatter}

\section*{Availability of Supporting Data}
More information about this project can be found at: \url{https://github.com/ericearl/hcp2nipype-hack2015/}. Further data and files supporting this project are hosted in the \emph{GigaScience} repository REFXXX.

\section*{Competing interests}
None

\section*{Author's contributions}
EE wrote the report, EE and all other authors wrote the software.

\section*{Acknowledgements}
The authors would like to thank the organizers and attendees of the 2015
OHBM Hackathon.

  
  
%%%%%%%%%%%%%%%%%%%%%%%%%%%%%%%%%%%%%%%%%%%%%%%%%%%%%%%%%%%%%
%%                  The Bibliography                       %%
%%                                                         %%
%%  Bmc_mathpys.bst  will be used to                       %%
%%  create a .BBL file for submission.                     %%
%%  After submission of the .TEX file,                     %%
%%  you will be prompted to submit your .BBL file.         %%
%%                                                         %%
%%                                                         %%
%%  Note that the displayed Bibliography will not          %%
%%  necessarily be rendered by Latex exactly as specified  %%
%%  in the online Instructions for Authors.                %%
%%                                                         %%
%%%%%%%%%%%%%%%%%%%%%%%%%%%%%%%%%%%%%%%%%%%%%%%%%%%%%%%%%%%%%

% if your bibliography is in bibtex format, use those commands:
\bibliographystyle{bmc-mathphys} % Style BST file
\bibliography{brainhack-report} % Bibliography file (usually '*.bib' )

\end{backmatter}
\end{document}
